\documentclass[11pt, a4paper]{article}
\usepackage[margin=1in]{geometry}
\usepackage{amsmath}
\renewcommand{\linespread}{1.0}

\title{IVR Assignment Report}
\author{Humbarwadi Srihari and Thanakrit Anutrakulchai}

\begin{document}
    \maketitle
    \section{Contributions}



    \section{Notation}
    For this report, we use the symbols
    $\vartheta$, $\nu$, $\phi$, $\psi$ to denote the value in radians of the angles 
    of joints 1, 2, 3, and 4, respectively. 
    We denote the position in 3D space of the $i^{th}$ joint with symbols
    $x_i$, $y_i$, $z_i$, and $x_e$, $y_e$, $z_e$ for the position of the end-effector,
    from any arbitrary left-handeded co-ordinate frame.
    % TODO is this true? %
    (We only use the difference between these values so the position of the frame
    does not matter. We choose a left-hand frame as that made getting positional data
    from the cameras more straightforward.)
    We also use the symbols $L_1$, $L_2$, $L_3$, $L_4$ for the
    lengths of the respective links. 
    We use short-hand notation for sine and cosine,
    as in $s_{\gamma}$ is to be read as $\sin(\gamma)$ and $c_{\gamma}$
    is to be read as $\cos(\gamma)$ for angles $\gamma$.
    



    \section{Joint State Estimation}
    \subsection{Part I \textemdash Fixing Joint 1}
    To calculate the angles of the joints, we first use blob detection to
    find where the joints are in 3D space. % TODO Elaborate?
    We use thresholding to filter out the positions of the green,
    yellow, blue, and red blobs which correspond to joints 1, 2, 3, 4 and 
    the end-effector, respectively. As we have two cameras, one facing
    the yz-plane, and the other facing the xz-plane, we perform this
    operation on the images received from both camera, and combine their
    results. % TODO Elaborate? %

    Once we have the necessary joints positions, we then consider the
    orientation of the links between them, specifically link 3 and link 4.
    To do so, we calculate unit-length vectors $\textbf{r}_3(\textbf{q})$,
    $\textbf{r}_4(\textbf{q})$ pointing from the center of
    joint 2 to joint 3, and joint 3 to joint 4, respectively. We choose
    two right-handed co-ordinate frames positioned at the base of the vectors
    so when all joints are at angle 0, the unit vectors have co-ordinates
    [0 0 1] in those frames.  
    These vectors are rotated as the joint angles change as so: % TODO explain? %


    $$
    \textbf{r}_3(\textbf{q}) :=
    \begin{pmatrix}
        r_{3x}\\
        r_{3y}\\
        r_{3z}
    \end{pmatrix} =
    \begin{pmatrix}
        c_{\phi}s_{\nu}\\
       -s_{\phi}\\
        c_{\phi}c_{\nu}
    \end{pmatrix} = \frac{1}{\sqrt{(x_4 - x_3)^2 + (y_4 - y_3)^2 + (z_3 - z_4)^2}}
    \begin{pmatrix}
        x_4 - x_3\\
        y_4 - y_3\\
        z_3 - z_4
    \end{pmatrix}
    $$

    $$
    \textbf{r}_4(\textbf{q}) :=
    \begin{pmatrix}
        r_{4x}\\
        r_{4y}\\
        r_{4z}
    \end{pmatrix} =
    \begin{pmatrix}
        c_{\nu}s_{\psi} + s_{\nu}c_{\phi}c_{\psi}\\
       -s_{\phi}c_{\psi}\\
       -s_{\nu}s_{\psi} + c_{\nu}c_{\phi}c_{\psi}
    \end{pmatrix} = \frac{1}{\sqrt{(x_e - x_4)^2 + (y_e - y_4)^2 + (z_4 - z_e)^2}}
    \begin{pmatrix}
        x_e - x_4\\
        y_e - y_4\\
        z_4 - z_e
    \end{pmatrix}
    $$

    The formulae above can be derived from geometry (e.g. as in the derivation
    of the spherical co-ordinates in SVCDE) or from multiplying the relevant
    rotation matrices to transform co-ordinate frames (translations are unnecessary
    as we are only concerned with orientation). Note that we negate the z component
    as we switch from a left-handed frame to right-handed frames.

    With these values, we can calculate the joint angles as so:
    \begin{align*}
        \nu  &= \arctan2(r_{3x}, r_{3z})\\
        \phi &= \arcsin(-r_{3y})\\
        \psi &= \arcsin(r_{4x}c_{\nu} - r_{4z}s_{\nu})
    \end{align*}

    where we calculate $\nu$ before $\psi$. 
    However, when $\phi$ is near $\pm\frac{\pi}{2}$, 
    $r_{3x}$ and $r_{3z}$ are near zero, so $\arctan2(r_{3x}, r_{3z})$
    oscillates quickly and rapidly over a short time. % TODO explain why?
    In such a case, we switch to calculating 
    $\nu = \arctan2(r_{4x}c_{\phi}c_{\psi} - r_{4z}s_{\psi}, \hspace{0.2cm}
                    r_{4x}s_{\psi} + r_{4z}c_{\phi}c_{\psi}
    )$, where we use the value of $\psi$ computed from the last iteration.
    Additionally, when $\psi$ is near $0$, as well, we stop calculating
    $\nu$ and keep the angle published constant. This is because we have
    a gimbal lock situation, any value of $\nu$ will produce the same observed
    set of positions, so we have no information on what $\nu$ can be, except
    that it cannot physically change from its previous value too much.


    \subsection{Part II \textemdash Fixing Joint 2}
    We follow much of the same methodology from Part I, but with a slight
    modification as most observed sets of positions of the joints now correspond
    with two possible set of angles. We calculate the unit-vectors now as:
    $$
    \textbf{r}_3(\textbf{q}) :=
    \begin{pmatrix}
        r_{3x}\\
        r_{3y}\\
        r_{3z} 
    \end{pmatrix} =
    \begin{pmatrix}
        s_{\vartheta}s_{\phi}\\
       -c_{\vartheta}s_{\phi}\\
        c_{\phi}
    \end{pmatrix} = \frac{1}{\sqrt{(x_4 - x_3)^2 + (y_4 - y_3)^2 + (z_3 - z_4)^2}}
    \begin{pmatrix}
        x_4 - x_3\\
        y_4 - y_3\\
        z_3 - z_4
    \end{pmatrix}
    $$

    $$
    \textbf{r}_4(\textbf{q}) :=
    \begin{pmatrix}
        r_{4x}\\
        r_{4y}\\
        r_{4z}
    \end{pmatrix} =
    \begin{pmatrix}
        c_{\vartheta}s_{\psi} + s_{\vartheta}s_{\phi}c_{\psi}\\
        s_{\vartheta}s_{\psi} - c_{\vartheta}s_{\phi}c_{\psi}\\
        c_{\phi}c_{\psi}
    \end{pmatrix} = \frac{1}{\sqrt{(x_e - x_4)^2 + (y_e - y_4)^2 + (z_4 - z_e)^2}}
    \begin{pmatrix}
        x_e - x_4\\
        y_e - y_4\\
        z_4 - z_e
    \end{pmatrix}
    $$

    and the angles as:
    \begin{align*}
        \vartheta &= \arctan2(r_{3x}sgn(\phi), -r_{3y}sgn(\phi))\\
        \phi      &= sgn(\phi)\arccos(\phi)\\
        \psi      &= \arcsin(r_{4x}c_{\vartheta} + r_{4z}s_{\vartheta})
    \end{align*}

    where the sign function $sgn(\phi)$ is $-1$ if $\phi < 0$, and $1$ otherwise.
    We use the sign of $\phi$ from the previous iteration for computing $\phi$. 
    Once again, we have problems with $\arctan2$ when $\phi$, 
    and hence $r_{3x}$, $r_{3y}$, is near 0. But, now we also have problems 
    with $sgn(\phi)$ giving the wrong value 
    (i.e. when $\phi$ crosses 0 or oscillates near 0). % TODO explain more? 
    In such cases, we compute 
    $\phi = \arcsin(r_{3x}s_{\vartheta} - r_{3y}c_{\vartheta})$,
     and $\vartheta = \arctan2(
         r_{4x}s_{\phi}c_{\psi} + r_{4y}s_{\psi}, \hspace{0.2cm}
         r_{4x}s_{\psi} - r_{4y}s_{\phi}c_{\psi} 
     )$. 
     Additionally, when $\psi$ is also near zero, we have aa
     gimbal lock situation, so we keep $\vartheta$, and thus also $\phi$,
     constant during these periods.

     Earlier, we noted that a set of joint positions, in general, corresponds
     to two possible sets of angles. The choice of $sgn(\phi)$ being $-1$ or $1$
     at the start picks which two of these sets we first compute. Once we have 
     a set, the set computed in the next iteration is the one consistent with
     the previously computed set, in that the angle values do not jump discontinuously.
     There is an exception to this: when $\vartheta$ crosses between $\pi$ and $-\pi$.
     In this scenario, we realize that we started with the wrong set of angles,
     and so we switch to the other set by adding or subtracting $\pi$ from $\vartheta$
     as appropriate, and negating the other angles.




    \section{Control}
    \subsection{Forward Kinematics}
    We derived Forward Kinematics for the robot using the D-H convention.
    \begin{center}
        \begin{tabular}{ |c|c|c|c|c| }
            \hline
            link & $\alpha$ & a & d & $\theta$\\
            \hline
            1 & $-\frac{\pi}{2}$ & 0 & $L_1$ & $\vartheta$\\
            2 & $-\frac{\pi}{2}$ & 0 & 0 & $-\frac{\pi}{2}$\\
            3 & $\frac{\pi}{2}$ & $L_3$ & 0 & $\phi$\\
            4 & 0 & $L_4$ & 0 & $\psi$\\
            \hline
        \end{tabular}
    \end{center}
    

    $$
    K(\textbf{q}) = 
        \begin{pmatrix}
            c_{\vartheta}s_{\psi}L_4 + s_{\vartheta}s_{\phi}c_{\psi}L_4 + s_{\vartheta}s_{\phi}L_3\\
            s_{\vartheta}s_{\psi}L_4 - c_{\vartheta}s_{\phi}c_{\psi}L_4 - c_{\vartheta}s_{\phi}L_3\\
            c_{\phi}c_{\psi}L_4 + c_{\phi}L_3 + L_1
        \end{pmatrix}
    $$
    \subsection{Inverse Kinematics}
    $$
    J(\textbf{q}) =
        \begin{pmatrix}
           -s_{\vartheta}s_{\psi}L_4 + c_{\vartheta}s_{\phi}c_{\psi}L_4 + c_{\vartheta}s_{\phi}L_3 & 
            s_{\vartheta}c_{\phi}c_{\psi}L_4 + s_{\vartheta}c_{\phi}L_3 & 
            c_{\vartheta}c_{\psi}L_4 - s_{\vartheta}s_{\phi}s_{\psi}L_4\\

            c_{\vartheta}s_{\psi}L_4 + s_{\vartheta}s_{\phi}c_{\psi}L_4 + s_{\vartheta}s_{\phi}L_3 &
           -c_{\vartheta}c_{\phi}c_{\psi}L_4 - c_{\vartheta}c_{\phi}L_3 &
            s_{\vartheta}c_{\psi}L_4 + c_{\vartheta}s_{\phi}s_{\psi}L_4\\

            0 &
           -s_{\phi}c_{\psi}L_4 - s_{\phi}L_3 &
           -c_{\phi}s_{\psi}L_4\\
        \end{pmatrix}
    $$

\end{document}